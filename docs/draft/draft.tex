% Options for packages loaded elsewhere
\PassOptionsToPackage{unicode}{hyperref}
\PassOptionsToPackage{hyphens}{url}
%
\documentclass[
  man]{apa6}
\usepackage{amsmath,amssymb}
\usepackage{lmodern}
\usepackage{iftex}
\ifPDFTeX
  \usepackage[T1]{fontenc}
  \usepackage[utf8]{inputenc}
  \usepackage{textcomp} % provide euro and other symbols
\else % if luatex or xetex
  \usepackage{unicode-math}
  \defaultfontfeatures{Scale=MatchLowercase}
  \defaultfontfeatures[\rmfamily]{Ligatures=TeX,Scale=1}
\fi
% Use upquote if available, for straight quotes in verbatim environments
\IfFileExists{upquote.sty}{\usepackage{upquote}}{}
\IfFileExists{microtype.sty}{% use microtype if available
  \usepackage[]{microtype}
  \UseMicrotypeSet[protrusion]{basicmath} % disable protrusion for tt fonts
}{}
\makeatletter
\@ifundefined{KOMAClassName}{% if non-KOMA class
  \IfFileExists{parskip.sty}{%
    \usepackage{parskip}
  }{% else
    \setlength{\parindent}{0pt}
    \setlength{\parskip}{6pt plus 2pt minus 1pt}}
}{% if KOMA class
  \KOMAoptions{parskip=half}}
\makeatother
\usepackage{xcolor}
\usepackage{graphicx}
\makeatletter
\def\maxwidth{\ifdim\Gin@nat@width>\linewidth\linewidth\else\Gin@nat@width\fi}
\def\maxheight{\ifdim\Gin@nat@height>\textheight\textheight\else\Gin@nat@height\fi}
\makeatother
% Scale images if necessary, so that they will not overflow the page
% margins by default, and it is still possible to overwrite the defaults
% using explicit options in \includegraphics[width, height, ...]{}
\setkeys{Gin}{width=\maxwidth,height=\maxheight,keepaspectratio}
% Set default figure placement to htbp
\makeatletter
\def\fps@figure{htbp}
\makeatother
\setlength{\emergencystretch}{3em} % prevent overfull lines
\providecommand{\tightlist}{%
  \setlength{\itemsep}{0pt}\setlength{\parskip}{0pt}}
\setcounter{secnumdepth}{-\maxdimen} % remove section numbering
% Make \paragraph and \subparagraph free-standing
\ifx\paragraph\undefined\else
  \let\oldparagraph\paragraph
  \renewcommand{\paragraph}[1]{\oldparagraph{#1}\mbox{}}
\fi
\ifx\subparagraph\undefined\else
  \let\oldsubparagraph\subparagraph
  \renewcommand{\subparagraph}[1]{\oldsubparagraph{#1}\mbox{}}
\fi
\newlength{\cslhangindent}
\setlength{\cslhangindent}{1.5em}
\newlength{\csllabelwidth}
\setlength{\csllabelwidth}{3em}
\newlength{\cslentryspacingunit} % times entry-spacing
\setlength{\cslentryspacingunit}{\parskip}
\newenvironment{CSLReferences}[2] % #1 hanging-ident, #2 entry spacing
 {% don't indent paragraphs
  \setlength{\parindent}{0pt}
  % turn on hanging indent if param 1 is 1
  \ifodd #1
  \let\oldpar\par
  \def\par{\hangindent=\cslhangindent\oldpar}
  \fi
  % set entry spacing
  \setlength{\parskip}{#2\cslentryspacingunit}
 }%
 {}
\usepackage{calc}
\newcommand{\CSLBlock}[1]{#1\hfill\break}
\newcommand{\CSLLeftMargin}[1]{\parbox[t]{\csllabelwidth}{#1}}
\newcommand{\CSLRightInline}[1]{\parbox[t]{\linewidth - \csllabelwidth}{#1}\break}
\newcommand{\CSLIndent}[1]{\hspace{\cslhangindent}#1}
\ifLuaTeX
\usepackage[bidi=basic]{babel}
\else
\usepackage[bidi=default]{babel}
\fi
\babelprovide[main,import]{english}
% get rid of language-specific shorthands (see #6817):
\let\LanguageShortHands\languageshorthands
\def\languageshorthands#1{}
% Manuscript styling
\usepackage{upgreek}
\captionsetup{font=singlespacing,justification=justified}

% Table formatting
\usepackage{longtable}
\usepackage{lscape}
% \usepackage[counterclockwise]{rotating}   % Landscape page setup for large tables
\usepackage{multirow}		% Table styling
\usepackage{tabularx}		% Control Column width
\usepackage[flushleft]{threeparttable}	% Allows for three part tables with a specified notes section
\usepackage{threeparttablex}            % Lets threeparttable work with longtable

% Create new environments so endfloat can handle them
% \newenvironment{ltable}
%   {\begin{landscape}\centering\begin{threeparttable}}
%   {\end{threeparttable}\end{landscape}}
\newenvironment{lltable}{\begin{landscape}\centering\begin{ThreePartTable}}{\end{ThreePartTable}\end{landscape}}

% Enables adjusting longtable caption width to table width
% Solution found at http://golatex.de/longtable-mit-caption-so-breit-wie-die-tabelle-t15767.html
\makeatletter
\newcommand\LastLTentrywidth{1em}
\newlength\longtablewidth
\setlength{\longtablewidth}{1in}
\newcommand{\getlongtablewidth}{\begingroup \ifcsname LT@\roman{LT@tables}\endcsname \global\longtablewidth=0pt \renewcommand{\LT@entry}[2]{\global\advance\longtablewidth by ##2\relax\gdef\LastLTentrywidth{##2}}\@nameuse{LT@\roman{LT@tables}} \fi \endgroup}

% \setlength{\parindent}{0.5in}
% \setlength{\parskip}{0pt plus 0pt minus 0pt}

% Overwrite redefinition of paragraph and subparagraph by the default LaTeX template
% See https://github.com/crsh/papaja/issues/292
\makeatletter
\renewcommand{\paragraph}{\@startsection{paragraph}{4}{\parindent}%
  {0\baselineskip \@plus 0.2ex \@minus 0.2ex}%
  {-1em}%
  {\normalfont\normalsize\bfseries\itshape\typesectitle}}

\renewcommand{\subparagraph}[1]{\@startsection{subparagraph}{5}{1em}%
  {0\baselineskip \@plus 0.2ex \@minus 0.2ex}%
  {-\z@\relax}%
  {\normalfont\normalsize\itshape\hspace{\parindent}{#1}\textit{\addperi}}{\relax}}
\makeatother

% \usepackage{etoolbox}
\makeatletter
\patchcmd{\HyOrg@maketitle}
  {\section{\normalfont\normalsize\abstractname}}
  {\section*{\normalfont\normalsize\abstractname}}
  {}{\typeout{Failed to patch abstract.}}
\patchcmd{\HyOrg@maketitle}
  {\section{\protect\normalfont{\@title}}}
  {\section*{\protect\normalfont{\@title}}}
  {}{\typeout{Failed to patch title.}}
\makeatother

\usepackage{xpatch}
\makeatletter
\xapptocmd\appendix
  {\xapptocmd\section
    {\addcontentsline{toc}{section}{\appendixname\ifoneappendix\else~\theappendix\fi\\: #1}}
    {}{\InnerPatchFailed}%
  }
{}{\PatchFailed}
\keywords{keywords\newline\indent Word count: X}
\DeclareDelayedFloatFlavor{ThreePartTable}{table}
\DeclareDelayedFloatFlavor{lltable}{table}
\DeclareDelayedFloatFlavor*{longtable}{table}
\makeatletter
\renewcommand{\efloat@iwrite}[1]{\immediate\expandafter\protected@write\csname efloat@post#1\endcsname{}}
\makeatother
\usepackage{lineno}

\linenumbers
\usepackage{csquotes}
\ifLuaTeX
  \usepackage{selnolig}  % disable illegal ligatures
\fi
\IfFileExists{bookmark.sty}{\usepackage{bookmark}}{\usepackage{hyperref}}
\IfFileExists{xurl.sty}{\usepackage{xurl}}{} % add URL line breaks if available
\urlstyle{same} % disable monospaced font for URLs
\hypersetup{
  pdftitle={The title},
  pdfauthor={First Author1 \& Ernst-August Doelle1,2},
  pdflang={en-EN},
  pdfkeywords={keywords},
  hidelinks,
  pdfcreator={LaTeX via pandoc}}

\title{The title}
\author{First Author\textsuperscript{1} \& Ernst-August Doelle\textsuperscript{1,2}}
\date{}


\shorttitle{Title}

\authornote{

Add complete departmental affiliations for each author here. Each new line herein must be indented, like this line.

Enter author note here.

The authors made the following contributions. First Author: Conceptualization, Writing - Original Draft Preparation, Writing - Review \& Editing; Ernst-August Doelle: Writing - Review \& Editing.

Correspondence concerning this article should be addressed to First Author, Postal address. E-mail: \href{mailto:my@email.com}{\nolinkurl{my@email.com}}

}

\affiliation{\vspace{0.5cm}\textsuperscript{1} Wilhelm-Wundt-University\\\textsuperscript{2} Konstanz Business School}

\abstract{%
One or two sentences providing a \textbf{basic introduction} to the field, comprehensible to a scientist in any discipline.

Two to three sentences of \textbf{more detailed background}, comprehensible to scientists in related disciplines.

One sentence clearly stating the \textbf{general problem} being addressed by this particular study.

One sentence summarizing the main result (with the words ``\textbf{here we show}'' or their equivalent).

Two or three sentences explaining what the \textbf{main result} reveals in direct comparison to what was thought to be the case previously, or how the main result adds to previous knowledge.

One or two sentences to put the results into a more \textbf{general context}.

Two or three sentences to provide a \textbf{broader perspective}, readily comprehensible to a scientist in any discipline.
}



\begin{document}
\maketitle

\hypertarget{introduction}{%
\section{Introduction}\label{introduction}}

** big picture = social implications of online teaching and how that has changed in a post-pandemic world
** small picture = what we are doing in this language program and how can we improve to meet the needs and wants of the students in their new reality

\begin{itemize}
\item
  Address the general idea
\item
  Explain the study
\item
  Address our contribution
\end{itemize}

pero claro el esta lli (\textbf{lee2022factors?}) lkasla

\hypertarget{literature-review}{%
\section{Literature Review}\label{literature-review}}

The development of online education has been evolving over the past few decades, offering an alternative to deliver learning in a more convenient manner to individuals who could not be physically present in the traditional educational classroom. Of special interest is the research and development of language education across time and how online education has played a role in this development, as well as the implementation of communicative strategies within it. Research and technological advances allowed online language education to develop greatly, however, the Covid-19 pandemic appeared to be its hardest challenge to date. According to the National Center for Education Statistics (2022), during the fall of 2020 over 75 percent of undergraduate students (11.8 million) were enrolled in at least one online course, whereas 44 percent (7.0 million) of all undergraduate students exclusively enrolled in online courses. Compared to fall 2019, these numbers represent a 97 percent increase in undergraduates taking at least one online course (11.8 million vs.~6.0 million) and a 186 percent increase in undergraduate students enrolled exclusively in online courses (7.0 million vs.~2.4 million).

Online learning comes with a new set of challenges that are not found in a traditional classroom setting, especially when, for many, it was not an option to have class anywhere but virtually. Being forced to switch modalities had consequences, both from the teachers' and students' perspectives, as they were either not ready for the change or lacked many materials. Top Hat (2020) did a major survey on undergraduate students' and teachers' perceptions during the online emergency switch in US universities. Their findings within students taking a language class were that overall, the quality of instruction received was worse in comparison to the in-person learning setting. In the same way, a vast majority of students felt that the class experience was unengaging as they missed spending time with faculty and fellow students. Within this swift change to online settings, many students also found that their academic load was not only reduced but also experienced problems with using online tools.

Language attitudes are key in the language learning environment, as they can determine the success of a student (e.g., Thompson, 2021). A learner with a positive attitude toward a learning environment will more likely succeed than a learner who finds the environment difficult or disengaged. Specifically, the attitudes in online learning have been shown to play a key role as students are physically detached from the learning environment, and rely more on self-regulation abilities (Alqurashi, 2016; Horvat et al., 2015; Ke \& Kwak, 2013; Wurst et al., 2008). Within this framework, anxiety has been linked in several research studies (e.g., Teimouri et al., 2019) as an indicator of poor linguistic development. Interestingly, anxiety and the fear of the unknown is a feeling that fluctuates with empirical contact throughout the academic appointment. Furthermore, Webb et al.~(2014) studied the attitudes of students and teachers in a flipped online classroom of Chinese. Their results showed an incremental fluctuation in their perception. That is, they found distance education would not meet student expectations at first, but through the semester they found a growing acceptance that led to successful results by the end of a 15-week term. A similar psychological development is observed in Ushida's (2005) study which investigated the changes in students' attitudes and motivation over time in online language courses. Their results showed a similar evolution within the student's attitudes, as students reduced their anxiety level as the semester progressed, feeling more comfortable with the system as well as the teaching materials by the end of the educational term.
Past educational experiences and abilities acquired through these experiences have an effect on how students shape attitudes toward a new learning encounter (Saito et al.~2018), as well as the familiarity and complexity of the instruments that the course requires (Heckel \& Ringeisen, 2019), which might involve some training and assistance before the beginning of the online course. During the covid outbreak, students and educators lacked experience with online tools as well as time to prepare, Moser et al.~(2021), leading to an overall feeling of less effective education Top hat (2020), Hodges et al.~(2020), Lee (2021). While the immediate transition to online education seemed rushed in Ana Ruiz-Alonso-Bartol et al (YEAR) the stress levels of undergraduates diminished from the beginning to the end of the academic term as they became more familiar with the instructional materials, and therefore gained more experience with their newfound educational setting. Their study of US undergraduates found varied responses in which some students enjoyed the increased autonomy and self-paced learning opportunities offered by the new online format. Along the same line, a survey of university students found a common belief that online education helped them continue their education while still learning, observing, and improving the quality of teaching as their professors gained more experience with the teaching instruments Chakraborty et al.~(2020).
The outbreak of COVID-19 has caused online teaching and learning to evolve very rapidly during the past two years. This was a global phenomenon in the history of language education which had many educational consequences that students most likely will continue to experience when approaching online education. With a global decrease in Covid-19 cases, most US universities slowly began to switch to a pre-pandemic educational setting or blend hybrid classes. The goal of this paper is dual, first to determine if online education is still in demand among our undergraduate students, and to identify what are the attitudes towards it in a context where online education is no longer mandatory. Our second goal is to better understand how our new post-pandemic reality modulates tendencies in university language class enrollment.

\hypertarget{research-gap-and-theoretical-approach}{%
\section{Research gap and theoretical approach}\label{research-gap-and-theoretical-approach}}

Education mirrors society and its challenges, with Covid-19 as a global example of the adaptations that educators around the world had to implement in order to continue providing education to their students. In this sense, education can be understood as a continuum that has endlessly developed according to the socioeconomic changes that society experiences throughout history.

Our premise in this research study is to gather information from our current students who are in an educational environment that has been in continuous change for the past few years. The effects of this ever-changing learning environment have affected students and continue to affect the way they interact with and view their education. Post-pandemic effects, along with world socio-economic changes such as inflation and the need to work, have brought a new educational reality that affects tendencies toward language enrollment. To better understand these tendencies, we employed the lens of The Grounded Theory, as it allows a cyclical observation of a changing world. The Grounded Theory considers a changing environment that can evolve, becoming the researcher a constant observer of the changes produced (Hernández Carrera, 2014). Having a better understanding of the constant change of tendencies that affect our students can help us not only to provide them with adequate tools to succeed but also help us in providing courses that match their needs and expectations.

This theoretical approach allowed us to perform mixed models including qualitative, quantitative, and open questionary deriving the hypothesis we present in this paper. This approach was chosen as it allows the researcher to observe a constant change in tendencies and derive a hypothesis based on the collected data. Due to the geographical limitations, we adopt a substantive Grounded Theory that helps explain the tendencies seen within our community of students and the factors that affect them (Corbin \& Strauss, 2014).

\hypertarget{current-study}{%
\section{Current Study}\label{current-study}}

Our premise in this research study is to gather information from our current students who are in an educational environment that has been in continuous change for the past few years.
The effects of this ever-changing learning environment have affected students and continue to affect modulating the way they interact with and view their education. Post-pandemic effects, along with world socio-economic changes, have brought a new educational reality that affects tendencies toward enrollment.
Having a better understanding of the constant change of tendencies that affect our students can help us not only in providing them with adequate tools to succeed but also help us in providing courses that match their needs and expectations.

\hypertarget{methods}{%
\section{Methods}\label{methods}}

\hypertarget{rqs}{%
\subsection{RQs}\label{rqs}}

The current study aims to answer the following research questions:

\begin{itemize}
\item
  How effective are online classes according to self-reported measures of language abilities?
\item
  How does our new post-pandemic reality modulate tendencies in language enrollment (in person vs online)?
\end{itemize}

We predict X based on Y.

In order to answer the research questions, a total of X students completed an anonymous survey. The following is a description of the participants that took part in the study and the task that they completed.

\hypertarget{participants}{%
\subsection{Participants}\label{participants}}

Students who were taking in person and online Spanish language classes at a Big 10 northeastern university were asked to complete the survey.
A total number of 87 people responded (update this with final count from Qualtrics).
The data from 6 surveys were removed due to incomplete responses.
The participants were compensated with extra credit towards their homework grades in their Spanish courses for completing the survey.
They were all typical university aged students (ages from 18-25) except one student, who belongs to the 25-35 age range.
{[}I don't know if this will be relevant, probably not, but we could include: 35 of the students were first years, 27 were second years, 11 were third years, and 9 were fourth years.{]}
Of the \# participants who completed the survey, 23 were taking a fully asynchronous online Spanish class, 51 were taking an in person class, and 7 were taking a hybrid course.

We also collected information about their linguistic backgrounds.
Of the \# of participants, 37 of the students speak another language besides English and Spanish.
The participants reported learning Spanish between the ages of X and X.
They had reported learning Spanish in X setting.
X number of participants had taken X number of years of Spanish in a high school setting.
X students have never taken Spanish courses prior to this course. (look up this information in the survey)

Additionally, we collected information about their work and commuting lives of these participants to be able to make judgements about how these factors impact learning attitudes.
The students ranged from living on campus (32 students) to living relatively close to campus 5-15 miles away (17 students), to living 15+ miles away from campus and having a long commute (33 students).
As for their working schedules, 51 students reported to not be currently working while attending university while 31 students are currently studying and working at the same time.
Of those 31 students who are working, five work between zero and five hours a week, ten work between five to ten hours a week, ten work between ten to fifteen hours and week and six work more than fifteen hours a week.

\hypertarget{material-and-procedure}{%
\subsection{Material and Procedure}\label{material-and-procedure}}

The survey was created on Qualtrics.
It was distributed via an online anonymous link to undergraduate students taking Spanish courses in a large university located in the northeast United States and it took an average of \# minutes to complete (find this info on qualtrics).
In total, the participants answered 46 questions.
The questions were divided into 4 sections (background information, online learning tools/accessibility, attitudes, and enrollment) that each aimed to gather information about different factors that might impact the responses.
The background information section asked questions about their age, how long they have studied Spanish, how far they live away from campus, and if they are working.
The online learning tools/accessibility questions asked how comfortable they felt using the learning management systems that their courses are using and how readily accessible stable Wifi, a computer or tablet, and the software and other technologies are that are required for the courses that they are taking.
The attitudes section aimed to gather the participants' thoughts about learning languages in person and in online settings, their motivations for taking a language course, and how they feel their language abilities have progressed.
The last section, enrollment, asked questions about school-work balance and the reasons why they enroll in online courses.

The aforementioned sections of the survey followed a mixed-methods design to examine students' perspectives through an interaction of quantitative and qualitative questions.
By using open-ended and close-ended questionnaires we were able to collect, contrast and students' testimonies.
The mixed-methods design of this study allowed us to compare open-ended questions, with close-ended questions obtaining tendencies, preferences, as well as attitudes toward enrollment.
Close-ended quantitative questions were presented in a scale from 5 to 1 in the questionnaire, with a 5 being valued as ``very useful'' and a 1 as ``not very useful''.

\hypertarget{data-analysis}{%
\subsection{Data analysis}\label{data-analysis}}

We used R (Version 4.1.2; R Core Team, 2021) and the R-packages \emph{papaja} (Version 0.1.1; Aust \& Barth, 2022), and \emph{tinylabels} (Version 0.2.3; Barth, 2022) for all our analyses.

\hypertarget{results}{%
\section{Results}\label{results}}

\hypertarget{discussion}{%
\section{Discussion}\label{discussion}}

\begin{itemize}
\tightlist
\item
  Tie to RQs and big picture
\end{itemize}

\hypertarget{conclusion}{%
\section{Conclusion}\label{conclusion}}

\begin{itemize}
\item
  Big Picture
\item
  Limitations
\item
  Future directions
\end{itemize}

The results of this research study can help inform the decisions that administrations at similar universities make about the courses that they offer in the future.

\newpage

\hypertarget{references}{%
\section{References}\label{references}}

\hypertarget{refs}{}
\begin{CSLReferences}{1}{0}
\leavevmode\vadjust pre{\hypertarget{ref-R-papaja}{}}%
Aust, F., \& Barth, M. (2022). \emph{{papaja}: {Prepare} reproducible {APA} journal articles with {R Markdown}}. Retrieved from \url{https://github.com/crsh/papaja}

\leavevmode\vadjust pre{\hypertarget{ref-R-tinylabels}{}}%
Barth, M. (2022). \emph{{tinylabels}: Lightweight variable labels}. Retrieved from \url{https://cran.r-project.org/package=tinylabels}

\leavevmode\vadjust pre{\hypertarget{ref-R-base}{}}%
R Core Team. (2021). \emph{R: A language and environment for statistical computing}. Vienna, Austria: R Foundation for Statistical Computing. Retrieved from \url{https://www.R-project.org/}

\end{CSLReferences}


\end{document}
